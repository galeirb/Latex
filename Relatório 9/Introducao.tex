\chapter*[Introdu\c{c}\~{a}o]{Introdu\c{c}\~{a}o}
\addcontentsline{toc}{chapter}{Introdu\c{c}\~{a}o}

	Os componentes utilizados na montagem deste experimento não fugiram dos padrões estudados em sala, foram utilizados:

\begin{itemize}
	\item $ 2N2222 $ (Transistor NPN)
	\item LED\footnote[1] {O LED é um componente eletrônico semicondutor, ou seja, um diodo emissor de luz ( L.E.D = Light emitter diode ), mesma tecnologia utilizada nos chips dos computadores, que tem a propriedade de transformar energia elétrica em luz. Tal transformação é diferente da encontrada nas lâmpadas convencionais que utilizam filamentos metálicos, radiação ultravioleta e descarga de gases, dentre outras. Nos LEDs, a transformação de energia elétrica em luz é feita na matéria, sendo, por isso, chamada de Estado sólido ( Solid State ). O LED é um componente do tipo bipolar, ou seja, tem um terminal chamado anodo e outro, chamado catodo. Dependendo de como for polarizado, permite ou não a passagem de corrente elétrica e, consequentemente, a geração ou não de luz. \cite{intro}
	} (comum)
\end{itemize}
	
	
	O objetivo do experimento é identificar a resistência necessária na junção emissora do diodo para que houvesse uma correte de $ 20mA $ correndo através do LED. 
