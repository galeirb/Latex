\chapter*[Introdu\c{c}\~{a}o]{Introdu\c{c}\~{a}o}
\addcontentsline{toc}{chapter}{Introdu\c{c}\~{a}o}
Neste relatório encontra-se o comparativo entre resultados teóricos e resultados experimentais de um transistor bipolar atuando como chave.
Tal uso é o mais simples de um transistor, e ele é caracterizado por possuir uma região de corte e uma região de saturação. A região de saturação é como se houvesse uma chave fechada do coletor para o emissor, e na região cortada é como uma chave aberta.
No experimento também verificou-se o transistor bipolar atuando como chave no acionamento de um relé.
A análise de erro foi realizada analisando medidas características do circuito, as suas tensões e intensidade de correntes e o quanto eram divergentes dos valores teóricos. 