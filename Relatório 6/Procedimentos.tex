\setcounter{topnumber}{5}
\setcounter{bottomnumber}{5}
\setcounter{totalnumber}{5}

\chapter{Procedimentos e resultados}

\centerline{\begin{minipage}[c]{\textwidth}
		\centering
		\noindent
		\captionof{figure}{Transistor bipolar atuando como chave}
		\includegraphics[width=0.5\textwidth]{Imagens/Figura1.jpg}
		\legend{Fonte: Produzido pelos autores}
		\label{Figura1}
\end{minipage}}


\begin{enumerate}
	\item Monte o circuito da Figura \ref{Figura1} com transistor bipolar atuando na condição de saturação. Meça as variáveis mostradas na Tabela \ref{Tabela1} e calcule os erros percentuais:

$$\%\; de\; erro = \frac{valor \; prático - valor \; teórico}{valor \; teórico} \times 100$$

\centerline{\begin{minipage}[c]{\textwidth}
		\centering
		\noindent
		\captionof{table}{Valores teóricos e práticos na condição de saturação}
		\begin{tabular}{cccc}
			\hline
			\textit{\textbf{Medida}} & \textit{\textbf{Valores teóricos}} & \textit{\textbf{Valores práticos}} & \textit{\textbf{Erro (\%)}} \\ \hline
			$ I_C $(SAT)                  & 13,5 $ mA $                            & 12,4 $ mA $                           &  8,15 $ \% $                       \\
			$ I_B $(SAT)                  & 1,43 $ mA $                           & 1,42 $ mA $                            & 0,7 $ \% $                        \\
			$ \beta_{CC} $(SAT)                 & 9,44                                & 8,73                             & 7,52 $ \% $                        \\
			$ V_{CE} $(SAT)                 & 0 $ V $                            & 413 $ mV $                            &                        
		\end{tabular}
		\legend{Fonte: Produzido pelos autores}
		\label{Tabela1}
\end{minipage}}

\centerline{\begin{minipage}[c]{\textwidth}
		\centering
		\noindent
		\captionof{table}{Valores teóricos e práticos na condição de corte}
		\begin{tabular}{cccccc}
			\toprule
			Variável & Valor teórico & Valor prático & Erro (\%) \\
			\midrule \midrule
			$I_{LED}$ &  $ A $ & $ 19,7 mA $ & 0 $ \% $ \\
			\midrule
			$I_{B}$ &  $ A$ & $ 1,066 mA $ & 0 $ \% $ \\
			\midrule
			$I_{C}$ &  $ A $ & $ 19,69 mA $ & 11,11 $ \% $ \\
			\midrule
			$V_{CE}$ &  $ V $ & $ 2,449 V $ & 11,11 $ \% $ \\
			\midrule
			$\beta$ & $ V $ & $ 18,47 $ & 11,11 $ \% $ \\
			\midrule
			\bottomrule
		\end{tabular}%
		\legend{Fonte: Produzido pelos autores}
		\label{Tabela5}
\end{minipage}}


	\item Monte o circuito da Figura \ref{Figura1} com o transitor bipolar atuando na condição de corte. Meça as variáveis mostradas na Tabela \ref{Tabela2} e calcle os erros percentuais.


	\centerline{\begin{minipage}[c]{\textwidth}
	\centering
	\noindent
	\captionof{table}{Valores teóricos e práticos na condição de corte}
\begin{tabular}{cccccc}
	\toprule
	Variável & Valor teórico & Valor prático & Erro (\%) \\
	\midrule \midrule
	$I_{LED}$ &  $ A $ & $ 19,7 mA $ & 0 $ \% $ \\
	\midrule
	$I_{B}$ &  $ A$ & $ 1,066 mA $ & 0 $ \% $ \\
	\midrule
	$I_{C}$ &  $ A $ & $ 19,69 mA $ & 11,11 $ \% $ \\
	\midrule
	$V_{CE}$ &  $ V $ & $ 2,449 V $ & 11,11 $ \% $ \\
	\midrule
	$\beta$ & $ V $ & $ 18,47 $ & 11,11 $ \% $ \\
	\midrule
	\bottomrule
\end{tabular}%
\legend{Fonte: Produzido pelos autores}
\label{Tabela2}
\end{minipage}}


	\item Verifique na folha de dados do transistor BC547B os valores de $ V_{CE} $ (SAT) $ V $ e de $ I_C $ (CORTE). Compare com os valores medidos.
	
Neste experimento usamos o transistor $ 2N2222 $, que de acordo como o datasheet, $ V_{CE} = 0,4 V$ e $ I_C = 0,01 \mu A$, onde no prático tivemos $ V_{CE} = 413 mV$ e $ I_C = 0 A$



	\item Monte o circuito da Figura \ref{TransistorBipolar} com transistor bipolar atuando como chave no acionamento de um relé. Verifique o correto funcionamento do circuito.
\end{enumerate}

	\centerline{\begin{minipage}[c]{\textwidth}
			\centering
			\noindent
			\captionof{figure}{Transistor bipolar atuando como chave no acionamento de um relé}
			\includegraphics[width=0.5\textwidth]{Imagens/TransistorBipolar.jpg}
		%	\legend{Fonte: Produzido pelos autores}
			\label{TransistorBipolar}
	\end{minipage}}
	
Para a simulação usamos o programa Proteus, montando o mesmo esquema da figura anterior, onde usamos um \textit{switch}  para ficamos variando a tensão de entrada na base de $ 0 V $ para $ 15 V $, ficando da seguinte maneira:

\centerline{\begin{minipage}[c]{\textwidth}
		\centering
		\noindent
		\captionof{figure}{Montagem do transistor bipolar atuando como chave no acionamento de um relé no Proteus}
		\includegraphics[width=0.5\textwidth]{Imagens/Imagem1.jpg}
		\legend{Fonte: Produzido pelos autores}
		\label{Imagem 1}
\end{minipage}}


Ao ligamos o circuito, com a base do transistor recebendo aproximadamente $ 0 V $, temos que não há variação no relé, fazendo com que o LED não receba nenhuma corrente, como mostrado na imagem abaixo:

\centerline{\begin{minipage}[c]{\textwidth}
		\centering
		\noindent
		\captionof{figure}{Etapa 1}
		\includegraphics[width=0.5\textwidth]{Imagens/Etapa1.jpg}
		\legend{Fonte: Produzido pelos autores}
		\label{Etapa 1}
\end{minipage}}

	
Agora mudando a tensão de entrada da base para $ 15 V $ temos que internamente, uma corrente circula pela bobina, fazendo criar um campo magnético, atraindo assim o contato do relé, fechando assim o circuito do LED, fazendo ele acender. Como mostramos na próxima figura:

\centerline{\begin{minipage}[c]{\textwidth}
		\centering
		\noindent
		\captionof{figure}{Etapa 2}
		\includegraphics[width=0.5\textwidth]{Imagens/Etapa2.jpg}
		\legend{Fonte: Produzido pelos autores}
		\label{Etapa 2}
\end{minipage}}

Temos assim, que quando cessamos a corrente da bobina, o contato do relé volta para a posição normal, abrindo assim o circuito do LED.

	
	

