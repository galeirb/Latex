\setcounter{topnumber}{5}
\setcounter{bottomnumber}{5}
\setcounter{totalnumber}{5}

\chapter{Desenvolvimento}
\section{Fundamentação Teórica}
\begin{enumerate}
	\item Diagrama de bode
	\item Resposta em frequência
	\item Explicar um filtro passa baixa, passa alta, rejeita faixa e passa faixa (um exemplo prático desta aplicação) \ldots
\end{enumerate}



\newpage
\section{Procedimentos}


\subsection{Amplificador $ 1^\circ $ estágio}

\centerline{\begin{minipage}[c]{\textwidth}
		\centering
		\noindent
		\captionof{figure}{Circuito elétrico do $ 1^\circ $ estágio do amplificador}
		\includegraphics[width=0.9\textwidth]{Imagens/Figura1.jpg}
		\legend{Fonte: Produzido pelos autores}
		\label{Figura1}
\end{minipage}}
	
\begin{enumerate}
	\item Dado o circuito da figura \ref{Figura1}, aplicar o gerador de funções com uma tensão $ 1,0mV $ e frequência de $ 1kHz $;
	\item Medir o ganho com o osciloscópio. Para tanto medir a tensão de saída $ V_s $ e de entrada $ V_e $.
	\item Usar o Bode Plotter (amplitude) e medir as frequências de corte e o ganho da banda passante em $ dB $ $ A $ $ db = 20 \log (V_s/V_e) $;
	\item Usar o Bode Plotter (fase) e medir os ângulos nas frequências importantes; Anotar todos os dados obtidos na tabela \ref{Tabela1}.
\end{enumerate}


\subsection{Amplificador $ 2^\circ $ estágio}

\centerline{\begin{minipage}[c]{\textwidth}
		\centering
		\noindent
		\captionof{figure}{Circuito elétrico do $ 2^\circ $ estágio do amplificador}
		\includegraphics[width=0.9\textwidth]{Imagens/Figura2.jpg}
		\legend{Fonte: Produzido pelos autores}
		\label{Figura2}
\end{minipage}}

\begin{enumerate}
	\item Dado o circuito figura \ref{Figura2}, aplicar o gerador de funções com uma tensão $ 1,0mV $ e frequência de $ 1kHz $;
	\item Medir o ganho com o osciloscópio. Para tanto medir a tensão de saída $ V_s $ e de entrada $ V_e $.
	\item Usar o Bode Plotter (amplitude) e medir as frequências de corte e o ganho da banda passante em $ dB $;
	\item Usar o Bode Plotter (fase) e medir os ângulos nas frequências importantes; Anotar todos os dados obtidos na tabela \ref{Tabela1}.
\end{enumerate}

\subsection{Amplificador com dois estágios}

\centerline{\begin{minipage}[c]{\textwidth}
		\centering
		\noindent
		\captionof{figure}{Circuito elétrico do amplificador com dois estágios}
		\includegraphics[width=0.9\textwidth]{Imagens/Figura3.jpg}
		\legend{Fonte: Produzido pelos autores}
		\label{Figura3}
\end{minipage}}

\begin{enumerate}
	\item Dado o circuito figura \ref{Figura3}, aplicar o gerador de funções com uma tensão $ 1,0mV $ e frequência de $ 1kHz $;
	\item Medir o ganho com o osciloscópio. Para tanto medir a tensão de saída $ V_s $ e de entrada $ V_e $.
	\item Usar o Bode Plotter (amplitude) e medir as frequências de corte e o ganho da banda passante em $ dB $;
\end{enumerate}

\subsection{Analise}
\begin{enumerate}
	\item Analise os resultados apontados na Tabela \ref{Tabela2} e explique:
	\begin{enumerate}
		\item Por que a frequência de corte inferior $ (fr_1) $ para o circuito \ref{Figura1} é maior que para o circuito \ref{Figura2}?
		\item Por que o ganho, para a faixa de frequência médias, do circuito \ref{Figura1} é bem maior do que o circuito \ref{Figura2}?
	\end{enumerate}
\end{enumerate}

\centerline{\begin{minipage}[c]{\textwidth}
		\centering
		\noindent
		%\captionof{table}{Valores $ CC $ teóricos e práticos}
		\begin{tabular}{ll|c|c|l|}
			\cline{3-5}
			&     & Circuito $ 1 $ & Circuito $ 1 $ & Circuito $ 2 $ \\ \hline
			\multicolumn{1}{|c|}{\multirow{3}{*}{Osciloscópio}} & $V_e$ $ (V_{pp}) $   &            &            &            \\ \cline{2-5} 
			\multicolumn{1}{|c|}{}                              & $V_s$ $ (V_{pp}) $   &            &            &            \\ \cline{2-5} 
			\multicolumn{1}{|c|}{}                              & $A_V$ $ (V_s/V_e) $   &            &            &            \\ \hline
			\multicolumn{1}{|l|}{\multirow{3}{*}{Bode Plotter}} & $A_V$ em frequências médias $ (dB) $   &            &            &            \\ \cline{2-5} 
			\multicolumn{1}{|l|}{}                              & frequência $ 1 $ a $ (-3dB) $ &            &            &            \\ \cline{2-5} 
			\multicolumn{1}{|l|}{}                              & frequência $ 2 $ a $ (-3dB) $ &            &            &            \\ \hline
		\end{tabular}
		\legend{Fonte: Produzido pelos autores}
		\label{Tabela1}
\end{minipage}}

\newpage
\section{Resultados}
Ao montar os circuitos \ref{Figura1},\ref{Figura2} e \ref{Figura3} no software MULTISIM, conseguimos obter os seguinte dados preenchidos na Tabela \ref{Tabela2}:	

\centerline{\begin{minipage}[c]{\textwidth}
		\centering
		\noindent
		\captionof{table}{Valores obtidos dos circuitos}
		\begin{tabular}{ll|c|c|l|}
			\cline{3-5}
			&     & Circuito $ 1 $ & Circuito $ 1 $ & Circuito $ 2 $ \\ \hline
			\multicolumn{1}{|c|}{\multirow{3}{*}{Osciloscópio}} & $V_e$ $ (V_{pp}) $   &            &            &            \\ \cline{2-5} 
			\multicolumn{1}{|c|}{}                              & $V_s$ $ (V_{pp}) $   &            &            &            \\ \cline{2-5} 
			\multicolumn{1}{|c|}{}                              & $A_V$ $ (V_s/V_e) $   &            &            &            \\ \hline
			\multicolumn{1}{|l|}{\multirow{3}{*}{Bode Plotter}} & $A_V$ em frequências médias $ (dB) $   &            &            &            \\ \cline{2-5} 
			\multicolumn{1}{|l|}{}                              & frequência $ 1 $ a $ (-3dB) $ &            &            &            \\ \cline{2-5} 
			\multicolumn{1}{|l|}{}                              & frequência $ 2 $ a $ (-3dB) $ &            &            &            \\ \hline
		\end{tabular}
		\legend{Fonte: Produzido pelos autores}
		\label{Tabela2}
\end{minipage}}

