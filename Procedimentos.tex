\setcounter{topnumber}{5}
\setcounter{bottomnumber}{5}
\setcounter{totalnumber}{5}

\chapter{Procedimentos e resultados}

\begin{enumerate}
	\item Monte o circuito da Figura 2 com transistor bipolar atuando como chave no acionamento de um relé. Verifique o correto funcionamento do circuito.
\end{enumerate}
	
	Add A imagem "TransistorBipolar"
	
Para a simulação usamos o programa Proteus, montando o mesmo esquema da figura anterior, onde usamos um \textit{switch}  para ficamos variando a tensão de entrada na base de $ 0 V $ para $ 15 V $, ficando da seguinte maneira:

	Add a imagem "Imagem 1"

Ao ligamos o circuito, com a base do transistor recebendo aproximadamente $ 0 V $, temos que não há variação no relé, fazendo com que o LED não receba nenhuma corrente, como mostrado na imagem abaixo:

	Add a imagem "Etapa 1"
	
Agora mudando a tensão de entrada da base para $ 15 V $ temos que internamente, uma corrente circula pela bobina, fazendo criar um campo magnético, atraindo assim o contato do relé, fechando assim o circuito do LED, fazendo ele acender. Como mostramos na próxima figura:

	Add a imagem "Etapa 2"
	
Temos assim, que quando cessamos a corrente da bobina, o contato do relé volta para a posição normal, abrindo assim o circuito do LED.
	
	

