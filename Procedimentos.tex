\setcounter{topnumber}{5}
\setcounter{bottomnumber}{5}
\setcounter{totalnumber}{5}

\chapter{Procedimentos e resultados}


	
	ADD a figura 1

\begin{enumerate}
	\item Monte o circuito da Figura 1 com transistor bipolar atuando na condição de saturação. Meça as variáveis mostradas na Tabela \ref{Tabela1} e calcule os erros percentuais:


	\centerline{\begin{minipage}[c]{\textwidth}
	\centering
	\noindent
	\captionof{table}{Valores teóricos e práticos na condição de saturação}
\begin{tabular}{cccc}
	\toprule
	Variável & Valor teórico & Valor prático & Erro (\%) \\
	\midrule \midrule
	$I_{C}$ (SAT) & 9,86 mA & 10,15 mA & 2,96 \\
	\midrule
	$I_{B}$ (SAT) & 30,42 $\mu A$ & 31,3 $\mu A$ & 2,87 \\
	\midrule
	$\beta_{cc}$ (SAT) & 324 &  324,28 & 0,09 \\
	\midrule
	$VCE$ (SAT) & 9,89 mA & 10,15 mA & 2,85 \\
	\midrule
	\bottomrule
\end{tabular}%
\legend{Fonte: Produzido pelos autores}
\label{Tabela1}
\end{minipage}}


	\item Monte o circuito da Figura 1 com o transitor bipolar atuando na condição de corte. Meça as variáveis mostradas na Tabela \ref{Tabela2} e calcle os erros percentuais.


	\centerline{\begin{minipage}[c]{\textwidth}
	\centering
	\noindent
	\captionof{table}{Valores teóricos e práticos na condição de corte}
\begin{tabular}{cccc}
	\toprule
	Variável & Valor teórico & Valor prático & Erro (\%) \\
	\midrule \midrule
	$I_{C}$ (CORTE) & 9,86 mA & 10,15 mA & 2,96 \\
	\midrule
	$I_{B}$ (CORTE) & 30,42 $\mu A$ & 31,3 $\mu A$ & 2,87 \\
	\midrule
	$VCE$ (SAT) & 9,89 mA & 10,15 mA & 2,85 \\
	\midrule
	\bottomrule
\end{tabular}%
\legend{Fonte: Produzido pelos autores}
\label{Tabela2}
\end{minipage}}


	\item Verifique na folha de dados do transistor BC547B os valores de $ V_{CE} $ (SAT) $ V $ e de $ I_C $ (CORTE). Compare com os valores medidos.




	\item Monte o circuito da Figura 2 com transistor bipolar atuando como chave no acionamento de um relé. Verifique o correto funcionamento do circuito.
\end{enumerate}

	
	Add A imagem "TransistorBipolar"
	
Para a simulação usamos o programa Proteus, montando o mesmo esquema da figura anterior, onde usamos um \textit{switch}  para ficamos variando a tensão de entrada na base de $ 0 V $ para $ 15 V $, ficando da seguinte maneira:

	Add a imagem "Imagem 1"

Ao ligamos o circuito, com a base do transistor recebendo aproximadamente $ 0 V $, temos que não há variação no relé, fazendo com que o LED não receba nenhuma corrente, como mostrado na imagem abaixo:

	Add a imagem "Etapa 1"
	
Agora mudando a tensão de entrada da base para $ 15 V $ temos que internamente, uma corrente circula pela bobina, fazendo criar um campo magnético, atraindo assim o contato do relé, fechando assim o circuito do LED, fazendo ele acender. Como mostramos na próxima figura:

	Add a imagem "Etapa 2"
	
Temos assim, que quando cessamos a corrente da bobina, o contato do relé volta para a posição normal, abrindo assim o circuito do LED.

	
	

